\section{List of Tenses and Moods}
\label{subsec:tense}
\textbf{Indicative Mood}
	\begin{itemize}[noitemsep]
		\item Simple Present (\textit{el presente simple})
		\item Past Perfect/Preterit (\textit{el pasado perfecto/el pret\'{e}rito })
		\item Imperfect (\textit{el imperfecto})
		\item Conditional (\textit{el condicional})
		\item Future (\textit{el futuro})
		\item Corresponding perfect tenses:
			\begin{itemize}[noitemsep]
				\item Present perfect (\textit{el presente perfecto})
				\item Preterit perfect (\textit{el pret\'{e}rito perfecto})
				\item Past perfect (\textit{el pasado perfecto})
				\item Conditional perfect (\textit{el condicional perfecto})
				\item Future perfect (\textit{el futuro perfecto})
			\end{itemize}
	\end{itemize}

\textbf{Subjunctive Mood}
	\begin{itemize}[noitemsep]
		\item Simple Present (\textit{el presente simple})
		\item Imperfect 1 \& 2 (\textit{el imperfecto 1 \& 2}) 
		\item Future (\textit{el futuro})
		\item Corresponding perfect tenses in the subjunctive:
			\begin{itemize}[noitemsep]
				\item Subjunctive present perfect (\textit{el presente perfecto del subjuntivo}) 
				\item Subjunctive past perfect 1 \& 2 (\textit{el pasado perfecto del subjuntivo 1 \& 2})
				\item Subjunctive conditional perfect (\textit{el condicional perfecto del subjuntivo})
				\item Subjunctive future perfect (\textit{el futuro perfecto del subjuntivo})
			\end{itemize}
	\end{itemize}

\textbf{Imperative Mood}
	\begin{itemize}[noitemsep]
		\item There's only really a present tense form. However, there are two sub-forms (not tenses):
			\begin{itemize}[noitemsep]
				\item Affirmative (\textit{afirmativo}): used to tell people to do something
				\item Negative (\textit{negativo}): used to tell people \underline{not} to do something
			\end{itemize}
	\end{itemize}

\section{Pronunciation and Spelling Rules}
\label{subsec:pronun}

Spanish is a pretty phonetic language, which is a blessing for students. This subsection isn't a comprehensive pronunciation or spelling guide, but it covers the reasons why we need to change the spelling of certain conjugated verbs. 

\subsection{C,G, and Z}
In this subsubsection, we're going to look into spelling changes related to \ita{-car}, \ita{-gar}, \ita{-zar}, and \ita{-guar} verbs . \\ 

In Spanish, the letter \ita{c} can be pronounced in two ways: like an English ``s'' or an English ``k.'' How do we know which pronunciation to use? It entirely depends on what letter follows the \ita{c}. 
\begin{itemize}[noitemsep]
	\item Consonant follows? Pronounce it like a ``k.''
	\item Vowel follows? It depends. Followed by an 
		\begin{itemize}[noitemsep]
			\item \ita{a, o} or \ita{u}? Pronounce it like a ``k.''
			\item \ita{e} or \ita{i}? Pronounce it like a ``s.''
		\end{itemize}
\end{itemize}

The exact same set of rules apply to the pronunciation of the letter \ita{g}. In Spanish, the letter \ita{g} can be pronounced either as a hard ``g'' sound (like in the word ``goat,'' written as /g/), or similar to the \ita{j} in Spanish (the \ita{jota}, written as /x/).\footnote{In linguistics, the former is a voiced velar fricative and the latter is a voiceless glottal/velar fricative, if you're curious.} Let's go through the same decision tree: 
\begin{itemize}[noitemsep]
	\item Consonant follows? Pronounce it like a /g/ 
	\item Vowel follows? It depends. Followed by an 
		\begin{itemize}[noitemsep]
			\item \ita{a, o} or \ita{u}? Pronounce it like a /g/
			\item \ita{e} or \ita{i}? Pronounce it like a /x/
		\end{itemize}
\end{itemize}

\begin{conf}{You'll be okay. }
To be honest, this tree can kinda seem intimidating at first. Most Spanish speakers don't use anything like this tree to talk. I highly recommend listening to native sources and repeating words and phrases to get more comfortable with pronunciation. This tree is just there to help you out in a moment of need, if you don't know how to pronounce a certain letter. I wouldn't really expect it to carry you through a conversation.
\end{conf}

Now that we've gone over that, what does it have to do with spelling changes?\\

Let's take the example verb \ita{buscar} (to look for), and we're going to conjugate it in the present subjunctive. 
First, let's look at the pronunciation of the \ita{c} in \ita{buscar}. Since the \ita{c} is followed by an \ita{a}, the \ita{c} will be pronounced like the English ``k.''i\\

Next, following the steps, we'd eventually wind up at \sout{\ita{busce}}. If you look at how the letter \ita{c} is pronounced in the fake word \sout{\ita{busce}}, you'll find that the \ita{c} would be pronounced as an English ``s.'' Uh-oh. If we conjugate it like this, the \ita{c}'s pronuncuation changes from the English ``k'' to an ``s.'' Therefore, the accepted fix is to replace the \ita{c} with a \ita{qu}, so that we get \ita{busque}, which maintains the hard \ita{c} sound. \\

The same thing takes place with \ita{-gar} verbs, where the accepted fix is to add a \ita{u} after the \ita{g} to maintain pronunciation.\\ 

The case of \ita{-zar} verbs is a bit different. In Spanish, we only use the letter \ita{z} when placing the letter \ita{c} would change the pronunciation. For example, the verb \ita{alcanzar} (to reach or to acheive) could \underline{not} be written as \sout{\ita{alcancar}}, since that would be pronounced as \ita{/alkankar/}. Hence, if we don't \underline{need} to use the \ita{z}, we use the \ita{c} instead. Therefore, the \ita{z} is replaced by the \ita{c} in \ita{-zar} verb subjunctive conjugations. The verb \ita{alcanzar} would therefore be conjugated as follows in the present subjunctive:\\



\conjugation[
	caption=Present subjunctive conjugation of \textit{alcanzar},
	label=verb:empezar
]{alcance \\ alcances \\ alcance \\ alcancemos \\ alcanc\'eis \\ alcancen}


Finally, let's talk about \ita{-guar} verbs. You don't really \ita{need} to know this, since these verbs aren't super common.\\

Remember when we added a \ita{u} after the letter \ita{g} in \ita{-gar} verbs to maintain that hard /g/ sound? Keep that in mind for the rest of this paragraph. In the string \ita{gu*}, the \ita{u} is pronounced conditionally:

\begin{itemize}[noitemsep]
	\item If the letter after \ita{gu} is \ita{a, o, u}, or a consonant the \ita{u} is pronounced. 
	\item Else, the \ita{u} is not pronounced. The only purpose it serves is to maintain the hard /g/ sound. 
\end{itemize}

In verbs like \ita{averiguar} (to find out), if we were to follow previously defined subjunctive conjugation rules, we'd get \sout{\ita{averigue}}. In the original verb \ita{averiguar}, the \ita{u} is indeed pronounced, while in the conjugated version we (incorrectly) generated, the \ita{u} would not be pronounced. Hence, the fix is to add a \ita{diaresis} to the \ita{u}, to indicate that we should pronounce it.\footnote{This breaks up what would have been a diphthong, creating a hiatus.} Hence, our final result should be \ita{averig{\"u}e}.

\subsection{Accent Rules}
\label{subsec:accents}
This is probably my \underline{favorite} aspect of the Spanish language. I really hope it becomes your favorite too, once you understand it.\\

\begin{conf}{Way to Approach This}
	Before we dive in, here's a quick bit of advice. This part of Spanish really exemplifies what language ``rules'' really are about. Most folks tend to think of them as super prescriptive and rigid laws that are to be followed at all costs, but that idea couldn't be more wrong. Language ``rules'' are really just a set of patterns that evolves just as the language itself evolves. In linguistics, the only thing that's truly \ita{wrong} would be something a native speaker would never say. What's wrong or right does not follow these rules; the rules follow the speakers. The beauty about Spanish pronunciation/spelling is that the spelling \ita{follows} the pronunciation - the written form evolved along with the spoken form.  
\end{conf}

Rant over; let's jump right in. In Spanish, accent marks can serve two purposes: they indicate which syllable should be stressed, or they're used to distinguish between words that would otherwise be written and said identically. \\

First, let's go over stress rules.\\

\begin{conf}{The Golden Rules}
If a word ends in either an \ita{n}, \ita{s}, or a vowel, the penultimate syllable of the word will be stressed. Else, the last syllable of the word is stressed. If the above two statements are not true for a word (meaning the stress falls elsewhere), the word will have an accent mark.
\end{conf}

Let's go through a couple of example words: \ita{alem{\'a}n} (German, m. s.) and \ita{alemana} (German, f. s.).\\

Let's first start out with \sout{\ita{aleman}}, a word that doesn't exist. In the masculine singular adjective for German, the stress falls on the vowel between the \ita{m} and the \ita{n}. However, since our fake word  \sout{\ita{aleman}} ends in an \ita{n}, the second-to-last syllable (the \ita{e}) will be stressed, which does \underline{not} match the spelling. Therefore, we'll need to add an accent mark on the last syllable, so that we get \ita{alem{\'a}n}. \\

Now let's move on to the feminine singular adjective for German, \ita{alemana}. The stress on this word falls on the same place as it did in the word \ita{alem{\'a}n}, the vowel between the \ita{m} and the \ita{n}. Since this word ends in a vowel, the stress already falls on the appropriate vowel (the \ita{a} between the \ita{m} and the \ita{n}) Hence, we don't need to add a diacritic to indicate where the stress falls; we already know since it follows the ``Golden Rules.''\\

Finally, let's go over a case where the addition of an accent mark is not for the purpose of changing pronunciation; rather, the accent mark just helps us distinguish between words that would otherwise be spelled identically. A subjunctive conjugation of the verb \ita{dar} (to give) is \ita{d{\'e}}. This is used to distinguish it from the preposition \ita{de} (of, from). Another example of this usage of an accent mark is in the word \ita{s{\'i}} (yes), which, without the accent mark, could be confused with \ita{si} (if).

\section{Apocapation}
\label{sec:apo}

Let's learn some fun linguistics stuff! \\

Apocapation is shortening the last bit of a word, and we do it pretty frequently in Spanish. In many cases, we just do it to \underline{certain} adjectives in the masculine singular form before nouns, but in some cases, we do it for the feminine singular form too, in the word \ita{grande} for instance.\\

\subsection {Masculine Singular Apocopes}
Let's look at the adjective \ita{primero} (first):
\begin{table}[ht]
\centering
\begin{tabular}[t]{lll}
\toprule
&Masculine&Feminine\\
\midrule
	Singular&\ita{primero}&\ita{primera}\\
	Plural& \ita{primeros}&\ita{primeras}\\
\bottomrule
\end{tabular}
	\caption{Inflections of \ita{primero}}
\end{table}

Therefore, the sentence ``The first car is the best,'' we would say \ita{El primer coche es el mejor}, instead of \sout{\ita{El primero coche es el mejor}}. \\

If you look at the table above, and the previous sentence, you might be wondering when we actually use the form \ita{primero}. We use that form when the adjective isn't actually placed before the noun. For example, in the sentence, ``He was the first to get up'' (\ita{Fue el primero en levantarse}), there's no noun after the adjective, so the long form, \ita{primero}, is used. We also use this long form in dates, such as to say ``the first of August'' (\ita{el primero de agosto}). Lastly, we use this form as an adverb, like when we say ``First, let's sit down'' (\ita{Primero, vamos a sentarnos}). \\

In the majority of cases, like for the word \ita{primero}, we only have to use the short form when we're describing masculine singular objects. The following is an exhaustive list of adjectives that only change in the \underline{masculine singular form}. \\

\begin{table}[H]
\centering
	\begin{tabular}[!htbp]{lll}
\toprule
Verb &Apocope& Meaning\\
\midrule
	\ita{uno} & \ita{un} & a/an \\
	\ita{alguno} & \ita{alg\'un} & some/any \\
	\ita{ninguno} & \ita{ning\'un} & no/none \\
	\ita{bueno}&\ita{buen}&good\\
	\ita{malo} & \ita{mal} & bad\\
	\ita{primero}& \ita{primer}& first\\
	\ita{tercero} & \ita{tercer} & third \\
	\ita{postrero} & \ita{postrer} & last \\

\bottomrule
\end{tabular}
	\caption{Exhaustive list of masculine singular apocopes}
\end{table}

\subsection{\ita{Grande}}
The adjective \ita{grande} works slightly differently from the apocopes we just saw, adjectives that only change in the masculine singular form. With \ita{grande} use the short adjective \ita{gran} in front of both masculine and feminine singular nouns. Therefore, we would say \ita{El Gran Juego} and \ita{una gran aventura}. \\

Remember, \ita{grande} means ``great'' when it's placed \underline{before} a noun. It means ``large'' when it's placed after a noun.


\subsection{\ita{Santo}}

This one's pretty easy. For male saints whose names do not begin with \ita{To-} or \ita{Do-}, use the short form \ita{San.} For male saints whose names begin with \ita{To-} or \ita{Do-}, use the long form \ita{Santo} \footnote{The reason for this is to avoid confusion. Saying \sout{\ita{San Tom\'as}} could sound like \sout{\ita{Santo mas}}.}. \\

Therefore, we'd say \ita{San Mart\'in} and \ita{San Miguel}, but we'd say \ita{Santo Tom\'as} or \ita{Santo Domingo.}

\subsection{\ita{Ciento}}
This one's also pretty easy. We use \ita{ciento} if it's followed by another number; for example, to say ``102 frogs'', we'd say \ita{ciento dos ranas}. Else, use the shortened form \ita{cien}. For example, we'd say \ita{cien ranas} (100 frogs) or \ita{cien mill\'on ranas} (100 million frogs). 

\subsection{\ita{Cualquiera}}

\ita{Cualquiera} is a pronoun that means ``anyone,'' in the sense of ``whichever human.'' For example, we could say \ita{Cualquiera puede cocinar} (Anyone can cook). \footnote{I feel oddly nostalgic and wish to watch Ratatouille for the millionth time.}  \\

\ita{Cualquier/Cualesquier}, the corresponding apocope, means ``any,'' in the sense of ``whichever one(s).'' For example, we could translate the sentence ``It doesn't really matter, you can pick whatever you want'' as \ita{Me da igual; puedes elegir cualquier cosa que quieras}. For instance, we could translate the sentence ``You can do it any way you like'' as \ita{Puedes hacerlo en cualquier forma},``You can do it in any way you like.''

\subsection{\ita{Tanto}}

Last, but not least, let's address the word \ita{tanto}, which means ``so much'' or ``so many.'' Remember that this word is typically used an adjective, so we tend to use it with reference to a given noun. Here are its inflections:  
\begin{table}[ht]
\centering
\begin{tabular}[t]{lll}
\toprule
&Masculine&Feminine\\
\midrule
	Singular&\ita{tanto}&\ita{tanta}\\
	Plural& \ita{tantos}&\ita{tantas}\\
\bottomrule
\end{tabular}
	\caption{Inflections of \ita{tanto}}
\end{table}

Hence, the following are all valid sentences:
\begin{enumerate}[noitemsep]
	\item \ita{Tengo tanto trabajo que hacer.} \arr ``I have so much work to do.''
	\item \ita{!`Hay tanta gente!} \arr ``There are so many people!''
	\item \ita{He cometido tantos errores.} \arr ``I've made so many mistakes.'
	\item \ita{Hay tantas maneras de resolver este problema.} \arr ``There are so many ways of solving this problem.''
\end{enumerate}

The word \ita{tanto} (this time unchanging) can also be used as an adverb, like in the sentence \ita{Te extra\~no tanto} (I miss you so much). \\

Let's now move on to the apocope, \ita{tan}. \ita{Tan} is the adverb, and means ``so.'' For instance, we could translate ``Frogs are so important to her'' as \ita{Las ranas son tan importantes para ella.}\\

Another use of both \ita{tanto} and \ita{tan} is in making comparisons. \\

We use the construction \ita{tan ... como} with adjectives and adverbs. For example, to translate the sentence ``Frogs are as cute as they are strong,'' we could say \ita{Las ranas son tan preciosas como fuertes.} \\

Using \ita{tanto} as an adverb, we use the construction \ita{tanto ... como} with verbs. For example, the sentence ``Frogs work as much as toads'' could be translated as \ita{Las ranas trabajan tanto como los sapos.}\footnote{I disagree, but that's beside the point. I'd argue that frogs are the more hard-working of the two.} \\

Finally, we use \ita{tanto} as an adjective (with nouns) in the construction \ita{tanto/a/os/as ... como}. For example, to translate the sentence \ita{Frogs have as many toes as toads}, we could say \ita{Las ranas tienen tantos dedos de pie como los sapos.}
