In this chapter, we're just going to talk about how to conjugate verbs in the subjunctive mood. We're not going to discuss when to use which forms; that's covered in later sections. For now, we shall conjugate like machines. \\

\section{Present Subjunctive}
To conjugate \textit{most} verbs in the present subjunctive, we can follow the following steps:
\begin{enumerate}[noitemsep]
	\item Find the \textit{yo} form of the verb in the present indicative
	\item Remove the \ita{-o} ending (leaving you with the ``stem'')
	\item Add the appropriate swapped ending in the present indicative
\end{enumerate}

What is this ``swapped ending?'' Remember how we have \textit{-ar}, \textit{-er}, and \textit{-ir} verbs? We're going to split those three categories into two categories, where all \textit{-ar} verbs make up their own category, and all \textit{-er} and \textit{-ir} verbs are clubbed into one category. When we use a ``swapped ending,'' it means that:
\begin{itemize}[noitemsep]
	\item If we have an \textit{-ar} verb, we conjugate it like an \textit{-er} verb in the present indicative
	\item If we have an \textit{-er} or an \textit{-ir} verb, we conjugate it like an \textit{-ar} verb in the present indicative
	\item If we're conjugating in the \textit{yo} form, just use the third-person singular form (the \textit{{\'{e}l, ella, usted}} form)
\end{itemize}

If that didn't make much sense, that's okay. Let's do a few examples, and then you might want to revisit the two previous paragraphs. They'll make a lot more sense after these examples: \\

Let's conjugate the verb \textit{tener} (to have) in the \textit{nosotr@s} form of the present subjunctive:
\begin{enumerate}[noitemsep]
	\item The \ita{yo} form of \ita{tener} is \ita{tengo}
	\item Let's remove the \ita{o}. We're left with \ita{teng-}
	\item Now we just need to add the right swapped ending. Since \ita{tener} is an \ita{-er} verb, we need to use the \ita{-ar} ending, which is \ita{-amos} in the \ita{nosotr@s} form. Now we have \ita{tengamos}, which is our final result. 
\end{enumerate}

\begin{conf}{@: \ita{La arroba}}
	For reference, the "at" sign (@, or the \ita{arroba}) is informally used to make terms gender neutral; don't use this for anything formal, like written work. In the previous example, I used \ita{nosotr@s} to avoid having to say \ita{nosotros} and \ita{nosotras}. The \ita{arroba} will be used for the rest of the guide to indicate gender neutrality.
\end{conf}

Let's conjugate the verb \ita{caer} (to fall) in the \ita{yo} form of the present subjunctive:
\begin{enumerate}[noitemsep]
	\item The \ita{yo} form of \ita{caer} is \ita{caigo}
	\item Let's remove the \ita{o}. We're left with \ita{caig-}
	\item Now we just need to add the right swapped ending. Since \ita{caer} is an \ita{-er} verb, we need to use the \ita{-ar} ending, which is \ita{-a} in the \ita{yo} form. Why is it \ita{-a} instead of \ita{-o}? It's because we're conjugating in the \ita{yo} form. You might want to take a look at bullet \#3 earlier in the chapter for clarification. Now we have \ita{caiga}, which is our final result. 
\end{enumerate}

Let's conjugate the verb \ita{atravesar} (to cross) in the \ita{ellos} form of the present subjunctive:
\begin{enumerate}[noitemsep]
	\item The \ita{yo} form of \ita{atravesar} is \ita{atravieso}
	\item Let's remove the \ita{o}. We're left with \ita{atravies-}
	\item Now we just need to add the right swapped ending. Since \ita{atravesar} is an \ita{-ar} verb, we need to use the \ita{-er} ending, which is \ita{-en} in the \ita{ellos} form. Now we have \ita{atraviesen}, which is our final result. 
\end{enumerate}

\begin{conf}{Stem Changes}
	For stem-changing verbs (in the present tense), we do \underline{not} change the stem in the \ita{nosotr@s} or \ita{vosotr@s} forms. 
\end{conf}

Hence, the full conjugation of \ita{atravesar} in the present subjunctive is as follows: 
\conjugation[
  caption=Present subjunctive conjugation of \textit{atravesar},
  label=verb:atravesar,
]{atraviese \\ atravieses  \\ atraviese  \\ atravesemos \\ atraves\'eis  \\ atraviesen}

\begin{conf}{Spelling Changes}
Something to keep in mind would be any spelling changes. In the preterit, remember how we had to change the spelling for \ita{-car}, \ita{-gar}, \ita{-zar}, and \ita{-guar} verbs? Well, those same changes apply in the present subjunctive. It's not as complicated as it seems, if you can remember that the Spanish language loves to keep its pronunciation regular, and will change the spelling of certain words to make sure the language is phonetic. It's actually a blessing, if you think about it. If you're curious about the reasons why there are spelling changes, there's a section about it in the \hyperref[subsec:pronun]{Pronunciation and Spelling Rules} part of the appendix. 
\end{conf}

Spelling change example:
Let's conjugate the verb \ita{pegar} (to hit or to glue) in the \ita{t{\'{u}}} form of the present subjunctive:
\begin{enumerate}[noitemsep]
	\item The \ita{yo} form of \ita{pegar} is \ita{pego}
	\item Let's remove the \ita{o}. We're left with \ita{peg-}
	\item Now we just need to add the right swapped ending. Since \ita{pegar} is an \ita{-ar} verb, we need to use the \ita{-er} ending, which is \ita{-es} in the \ita{t{\'{u}}} form. Now we have \sout{\ita{peges}}. Hey, wait a sec! That's not right! Since it's a \ita{-gar} verb, we need to add a \ita{u} after the \ita{g}. Now we have \ita{pegues}, which is our final result.
\end{enumerate}

We're not done yet, though. What we just covered is how we conjugate \ita{most} verbs. There are still some that march to their own beat: \\



\begin{table}[ht]
\centering
\begin{tabular}[t]{ll}
\toprule
\textbf{Verb} & \textbf{Stem} \\
\midrule
	\ita{saber} & \ita{sepa}\\
	\ita{estar} & \ita{est{\'{e}}}\\
	\ita{dar} & \ita{d{\'e}}\\
	\ita{haber} & \ita{haya}\\
	\ita{ser} & \ita{sea} \\
	\ita{ir} & \ita{vaya} \\
	\bottomrule
%\label{List of irregular verbs in the Present Subjunctive}

\end{tabular}
	\caption{{\label{tab:irrpres}}List of irregular verbs in the present subjunctive}
\end{table}

After getting the stem, you'll just need to tag on the ending. 
\begin{conf}{Accent Marks}
	For \ita{estar} (to be) and \ita{dar} (to give), you'll need to adjust the accent marks following the accent mark rules, which you can read up in the \hyperref[subsec:pronun]{Pronunciation and Spelling Rules} part of the appendix. If you don't want to read that, that's okay. I've left the conjugations anyway if you prefer to just memorize the conjugations for these two verbs.
\end{conf}

\conjugation[
  caption=Present subjunctive conjugation of \textit{estar},
  label=verb:estar,
]{est\'e \\ est\'es  \\ est\'e  \\ estemos \\ est\'eis \\ est\'en}

\conjugation[
	caption=Present subjunctive conjugation of \textit{dar},
  label=verb:dar,
]{d\'e \\ des  \\ d\'e  \\ demos \\ deis \\ den}





%%%%%%%%%%%%%%%%%%%%%%%%%%%%%%%%%%%%%%%%%%%%%%%%%%%%%%%%%%%%%%%%%%%%%%%%%%%%%


\section{Subjunctive Imperfect}

The next thing that we're going to go over are conjugations in the subjunctive imperfect. A cool thing about this verb tense is that there are \ita{zero} exceptions to the rules given below, which is pretty rare. \\

There are two sets of conjugations in the subjunctive imperfect, the \ita{-ra} (called the type 1 endings) or the \ita{-se} (called the type 2 endings). Within this section, I'm going to refer to them as the \ita{-ra} endings or the \ita{-se} endings for consistency. We pretty much exclusively use the \ita{-ra} forms in conversation, while in writing, both the \ita{-ra} and the \ita{-se} forms are used.\\

\subsection{Type 1 (\ita{-ra}) Endings}
Let's first talk about how we conjugate with \ita{-ra} endings: 
\begin{enumerate}[noitemsep]
	\item Find the \ita{ell@s, Uds.} form of the verb in the preterit
	\item Remove the \ita{-ron} ending (leaving you with the ``stem'')
	\item Add the appropriate ending given below. 
\end{enumerate}
\

\conjugation[
	caption=Type 1/(\ita{-ra}) endings in the subjunctive imperfect
]{-ra \\ -ras \\ -ra \\ \normalfont{Accent +} \ita{-ramos} \\ -rais \\ -ran}


What does ``Accent + \ita{-ramos}'' mean in the \ita{nosotr@s} form? Remember to add an accent mark to the last syllable of the stem before adding the ending. Why do we need to add an accent? It's to make sure that syllable remains stressed. A more in-depth explanation is given in the \hyperref[subsec:pronun]{Pronunciation and Spelling Rules} part of the appendix. \\

Let's conjugate a sample verb, \ita{poder} (to be able):
\begin{enumerate}[noitemsep]
	\item The \ita{ell@s, Uds.} form of \ita{poder} is \ita{pudieron}
	\item Let's remove the \ita{ron}. We're left with \ita{pudie-}
	\item The final step is to add the endings (see below). Pay attention to the \ita{nosotr@s} form. 
\end{enumerate}


\conjugation[
	caption=Conjugations of \ita{poder} in the subjunctive imperfect (\ita{-ra})
]{pudiera \\ pudieras \\ pudiera \\ pudi\'eramos \\ pudierais \\ pudieran}

Let's conjugate another verb, \ita{hablar} (to talk/speak) for funsies: 

\begin{enumerate}[noitemsep]
	\item The \ita{ell@s, Uds.} form of \ita{hablar} is \ita{hablaron}
	\item Let's remove the \ita{ron}. We're left with \ita{habla-}
	\item The final step is to add the endings (see below). Pay attention to the \ita{nosotr@s} form. 
\end{enumerate}

\conjugation[
	caption=Conjugations of \ita{hablar} in the subjunctive imperfect (\ita{-ra})
]{hablara \\ hablaras \\ hablara \\ habl\'aramos \\ hablarais \\ hablaran}

\subsection{Type 2 (\ita{-se}) Endings}

Now let's talk about how we conjugate with \ita{-se} endings. The process is fairly similar to conjugating with \ita{-ra} endings. The only thing that differs are the endings we add to the stem. \\

\begin{enumerate}[noitemsep]
	\item Find the \ita{ell@s, Uds.} form of the verb in the preterit
	\item Remove the \ita{-ron} ending (leaving you with the ``stem'')
	\item Add the appropriate ending given below. 
\end{enumerate}

\conjugation[
	caption=Type 2/(\ita{-se}) endings in the subjunctive imperfect
]{-se \\ -ses \\ -se \\ \normalfont{Accent +} \ita{-semos} \\ -seis \\ -sen}


We're going to conjugate the same two verbs (\ita{poder} and \ita{hablar}), but this time with the \ita{-se} endings. \\

Let's conjugate a sample verb, \ita{poder} (to be able):
\begin{enumerate}[noitemsep]
	\item The \ita{ell@s, Uds.} form of \ita{poder} is \ita{pudieron}
	\item Let's remove the \ita{ron}. We're left with \ita{pudie-}
	\item The final step is to add the endings (see below). Pay attention to the \ita{nosotr@s} form. 
\end{enumerate}

\conjugation[
	caption=Conjugations of \ita{poder} in the subjunctive imperfect (\ita{-se})
]{pudiese \\ pudieses \\ pudiese \\ pudi\'esemos \\ pudieseis \\ pudiesen}



Let's conjugate another verb, \ita{hablar} (to talk/speak) for funsies: 

\begin{enumerate}[noitemsep]
	\item The \ita{ell@s, Uds.} form of \ita{hablar} is \ita{hablaron}
	\item Let's remove the \ita{ron}. We're left with \ita{habla-}
	\item The final step is to add the endings (see below). Pay attention to the \ita{nosotr@s} form. 
\end{enumerate}

\conjugation[
	caption=Conjugations of \ita{hablar} in the subjunctive imperfect (\ita{-se})
]{hablase \\ hablases \\ hablase \\ habl\'asemos \\ hablaseis \\ hablasen}
%%%%%%%%%%%%%%%%%%%%%%%%%%%%%%%%%%%%%%%%%%%%%%%%%%%%%%%%%%%%%%%%%%%%%%%%%%%%%%%%%%%%%%%%%%%%%%%%%%%%%%%%%%%%%%%%%%%%%%%%
\section{The Perfect Tenses}

The subjunctive perfect tenses work in the same way that perfect tenses worked in the indicative. All conjugations take the form of \ita{haber} + past participle.\\ 

Here are the conjugations of the verb \ita{haber} in the subjunctive: 


	\begin{table}[H]
	\centering
	\begin{tabular}{lllll}
		\toprule
		& \textbf{Present} & \textbf{Imperfect: \ita{-ra}} &\textbf{ Imperfect: \ita{-se}} & \textbf{Future} \\
		\midrule
		\ita{\textbf{Yo}}  & \ita{haya} & \ita{hubiera} & \ita{hubiese} & \ita{hubiere} \\
		\ita{\textbf{T{\'u}}} & \ita{hayas} & \ita{hubieras} & \ita{hubieses} & \ita{hubieres}\\
		\ita{\textbf{{\'E}l, Ella, Ud.}} & \ita{haya} & \ita{hubiera} & \ita{hubiese} & \ita{hubiere}\\
		\ita{\textbf{Nosotr@s}} & \ita{hayamos} & \ita{hubi{\'e}ramos} & \ita{hubi{\'e}semos} & \ita{hubi{\'e}remos} \\
		\ita{\textbf{Vosotr@s}} & \ita{hay{\'a}is} & \ita{hubierais} & \ita{hubieseis} & \ita{hubiereis} \\
		\ita{\textbf{Ell@s, Uds.}} & \ita{hayan} & \ita{hubieran} & \ita{hubiesen} & \ita{hubieren}\\
		\bottomrule
	\end{tabular}
		\caption{{\label{tab:perfect}}Conjugations of \ita{haber} in the subjunctive}
\end{table}
%%%%%%%%%%%%%%%%%%%%%%%%%%%%%%%%%%%%%%%%%%%%%%%%%%%%%%%%%%%%%%%%%%%%%%%%%%%%%%%%%%%%%%%%%%%%%%%%%%%%%%%%%%%%%%%%%%%5%%%%
\section{The Subjunctive Future (Optional)}

This verb tense is used pretty rarely. You'll see it used exclusively in legal documents and other extremely formal pieces of literature, so it's okay if you skip this section and move on to the next chapter. However, if you're curious, read on! \\

This verb tense also won't be discussed for the rest of the guide, so I'm also going to talk about its usage as well as conjugation in this section. The subjunctive future is (sparingly) used to indicate uncertainty in the future, and is typically replaced by the present subjunctive in the vast majority of cases. \\

Conjugations are similar to what we saw in the subjunctive imperfect: 
\begin{enumerate}[noitemsep]
	\item Find the \ita{ell@s, Uds.} form of the verb in the preterit
	\item Remove the \ita{-ron} ending (leaving you with the ``stem'')
	\item Add the appropriate ending given below. 
\end{enumerate}

\conjugation[
	caption=Endings in the subjunctive future
]{-re \\ -res \\ -re \\ \normalfont{Accent +} -remos \\ -reis \\ -ren}

Let's conjugate an example verb, \ita{leer} (to read):
\begin{enumerate}[noitemsep]
	\item The \ita{ell@s, Uds.} form of \ita{leer} is \ita{leyeron}
	\item Let's remove the \ita{ron}. We're left with \ita{leye-}
	\item The final step is to add the endings (see below). Pay attention to the \ita{nosotr@s} form. 
\end{enumerate}

\conjugation[
	caption=Conjugations of \ita{leer} in the subjunctive future
]{leyere \\ leyeres \\ leyere \\ ley\'eremos \\ leyereis \\ leyeren}



