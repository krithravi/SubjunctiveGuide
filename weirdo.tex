\label{sec:weirdo}
In this chapter, we're going to talk about when to use the subjunctive mood. 


\begin{conf}{The Big Idea}
If you get nothing else from this guide, please at least walk away with this: the subjunctive mood is used to express uncertainty. Sometimes rules will require you to use the subjunctive, and sometimes the use of the subjunctive is left to the speaker's discretion. Regardless of how and/or why it is used, the subjunctive will add an element of uncertainty or doubt to the sentence. 
\end{conf}

 WEIRDO is a common acronym that's used to teach students when to use the subjunctive mood instead of the indicative. \\ 

As an overview, this is what WEIRDO stands for:
\begin{itemize}[noitemsep]
	\item W: Wishes
	\item E: Emotional Expressions
	\item I: Impersonal Expressions
	\item R: Recommendations
	\item D: Doubt \& Denial
	\item O: \ita{Ojal{\'a}}
\end{itemize}

There are typically two clauses in a sentence uses the subjunctive, a main clause that's in the indicative, and a subordinate clause that's in the subjunctive. If the main clause happens to be a WEIRDO clause, then there's really high chance that you'll need to use the subjunctive in the subordinate clause. In this section, we'll pretty much exclusively be dealing with cases where the main clause is a WEIRDO clause. \\

The tense of the main clause determines the tense of the subordinate clause: 
\begin{itemize}[noitemsep]
	\item If the main clause is in the present tense, the subordinate clause can be in the present subjunctive or in the present imperfect; it depends on when the event is supposed to happen
		\begin{itemize}[noitemsep]
			\item If the event is supposed to happen in the present or future, use the present subjunctive
			\item If the event was supposed to happen in the past, use the subjunctive imperfect
		\end{itemize}
	\item If the main clause is in the preterit/imperfect, the subordinate clause in the subjunctive imperfect
	\item If the main clause is in the conditional, the subordinate clause is in the subjunctive imperfect
	\item Following the above three statements: if the WEIRDO clause is in a perfect tense, choose the appropriate subjunctive perfect tense
\end{itemize}

Finally, these are the most common features of sentences that use the subjunctive:
\begin{itemize}[noitemsep]
	\item A relative pronoun, typically \ita{que} (what), but could also include \ita{quien} (who), \ita{donde} (where), or \ita{cuando} (when) \footnote{Don't confuse these with interrogative pronouns, like \ita{qu{\'e}} (what), \ita{qui{\'e}n} (who), \ita{d{\'o}nde} (where), or \ita{cu{\'a}ndo} (when), which are used to form questions. Relative pronouns are used to provide additional information. For instance, in the sentence, ``I'm looking for someone who speaks Arabic well,'' the word ``who'' is a relative pronouns that provides extra information about the individual. In Spanish, that same sentence would be translated as \ita{Busco a alguien que hable bien el {\'a}rabe}, taking the subjunctive. }
	\item There are two subjects (one for each clause)
	\item One WEIRDO clause, and a subordinate clause
\end{itemize}

For the rest of the guide, when I say that an expression``triggers'' the subjunctive, I mean that the expression requires the subjunctive to be used. \\

Don't worry if this stuff doesn't make that much sense right now. It'll make more sense after we've gone through a few examples in the subsequent sections. \\

Let's dive in! 

\begin{conf}{Get caffeinated!}
Seriously. If you're kinda tired right now, take a nap before reading this. If you're wide awake, you might want to pour yourself a cup of coffee, tea, or Diet Coke before reading the next few sections. This is the part that can and hopefully will mentally fatigue you. If you go through the next few sections without at least getting kinda confused, there are two possibilities: (a) you already know all of this, or (b) you understood none of this. 
\end{conf}
\section{W: Wishes}

One use of the subjunctive is to express wishes. For example, to translate the sentence ``I hope you're doing okay,'' we would need to use the subjunctive, in saying \ita{Espero que est{\'e}s bien}. Let's break it down, point by point:

\begin{enumerate}
	\item Let's rewrite the sentence in an equivalent form: ``I hope that you're doing okay.'' In English, relative pronouns are sometimes optional, while in Spanish they're always required when linking clauses. Adding in the optional relative pronoun helps us see the two clauses.
	\item The first clause, the ``I hope'' part, is the WEIRDO clause, since it expresses a wish. Wishes are things that we're \underline{not sure} about, which expresses uncertainty. Since wishes are uncertain, this tells us that we'll need to use the subjunctive at some point in this translation, but not just yet. Since the speaker is sure that they're not sure (that's a mind-boggle, but stick with me here), that part of the sentence takes the indicative. That's the reason why we say \ita{Espero que} in the first part of the translated sentence. 
	\item After the relative pronoun \ita{que}, we get the subordinate clause. The speaker isn't really sure if the person they're addressing is actually okay, which is why we need to use the subjunctive here. Now, we have \ita{est{\'e}s bien}.
	\item Piecing that together, we now have \ita{Espero que est{\'e}s bien.}
\end{enumerate}

Woo-hoo! One down, with 50 gazillion more to go! Don't worry. You'll be able to do all that reasoning in a split-second with more practice. \\

Let's do another example with wishes. This time, we're translating the sentence ``They hoped she would pass the test.'' The translation for that would be \ita{Esperaban que aprobabara el examen. }\\

\begin{enumerate}
	\item Since we can't really see the relative pronoun in our sentence, let's rewrite it: ``They hoped that she would pass the test.''
	\item The first clause, the ``They hoped,'' is the WEIRDO clause, since it expresses a wish. Since the speakers don't know whether she passed the test, they have to use the subjunctive. With the subjunctive, a key thing to remember is to put yourself in the speaker's/speakers' shoes, and see how \underline{they} would feel, certain or uncertain about the outcome. 
	\item Again, like in the previous example, the speakers are sure that they're not sure. That's why the WEIRDO clause is always in the indicative. Hence, so far, we have \ita{Esperaban que} as the first part of the translation.
	\item After the relative pronoun \ita{que}, we get the subordinate clause. The speakers aren't really sure if she's passed the exam, which is why the use of the subjunctive is necessary. Remember the bit in the beginning of this subsection that talked about how the tense of the WEIRDO clause determines the tense of the subordinate clause? That becomes important in this case. Since the WEIRDO clause is in the imperfect, the verb in the subordinate clause has to be conjugated in the subjunctive imperfect. Now we have \ita{aprobara el examen} as the latter half of the translation.
	\item Piecing both together, we have \ita{Esperaban que aprobara el examen.}
\end{enumerate}

But wait! What if the first half of the sentence is in the present, but we want the latter half to be in the past tense? Just use the perfect tense. For example, to translate the sentence ``I hope you had a good summer,'' we'd say \ita{Espero que hayas pasado un buen verano}. \\

Finally, let's work through an example that does \underline{not} use the subjunctive. Let's translate the sentence ``He hopes to get into med school.'' \\

Why shouldn't we use the subjunctive? Let's go through our checklist (the set of bullets towards the beginning of the chapter):
\begin{itemize}[noitemsep]
	\item Is there a relative pronoun? Sure, if we rewrite our sentence as ``He hopes that he'll get into med school.''
	\item Are there two subjects? \underline{No!} In many (not all) cases, not having two different subjects is a pretty good reason to not use the subjunctive.
\end{itemize}

Here's what \underline{incorrect} \footnote{In linguistics, the only things that are wrong are things that native speakers would never say. I rant about this in the \hyperref[subsec:accents]{Accent Rules} section in the appendix.} reasoning would have looked like: 
\begin{enumerate}
	\item Rewriting the sentence with the relative pronoun, we'd get ``He hopes that he'll get into med school.''
	\item Translating the WEIRDO clause, we'd have \ita{\'El espera que} as the first chunk.
	\item After the relative pronoun, we'll need to translate the subordinate clause. Now, we'd get \ita{entre a la escuela de medicina} as the latter half.
	\item Piecing those together, we'll get the incorrect translation \ita{\'El espera que entre a la escuela de medicina.}
\end{enumerate}

So what's wrong with that? While it may be ``grammatically correct,'' (whatever that means) it's wrong for pretty much all intents and purposes, since native speakers wouldn't be inclined to say that sentence we just produced. \\

When we don't have two unique subjects, native speakers prefer to use the infinitive instead. Hence, the correct translation would really be \ita{\'El espera entrar a la escuela de medicina} (lit. He hopes to get into med school.)\\

\subsection{Wishes with \ita{que}}

This is not the most common usage of the subjunctive, but you'll still see it occasionally. The best way to understand this usage might be to look at a few examples sentences:

\begin{enumerate}[noitemsep]
	\item ``May they find happiness.'' \arr \ita{Que encuentren felicidad.}
	\item ``May you go with God.'' \arr \ita{Que vayas con Dios.}
	\item ``May you live a hundred years!'' \arr \ita{!`Que vivas 100 a\~nos!}
\end{enumerate}

Notice the pattern? The formula \ita{que +} verb in the subjunctive is another way of saying these sorts of wishes. It can be helpful to think of it as a shortened form of \ita{\sout{Espero} que}. As you probably saw in the previous examples, these sorts of sentences aren't used often in everyday speech. 

We'll look into several more examples like these ones in the following sections, and you'll get a better feel for when to or when \underline{not} to use the subjunctive.
\section{E: Emotional Expressions}
%%%%%%%%%%%%%%%%%%%%%%%%%%%%%%%%%%%%%%%%%%%%%%%%%%%%%%%%%%%%%%%%%%%%%%%%%%%%%%%%%%%%%%%%%%%%%%%%%%%%%%
Emotional expressions with the subjunctive can be pretty challenging, just like expressions with doubt and denial. Emotional expressions express an opinion on an event, and are therefore subjective. That's why these expressions can trigger the subjunctive. \\

Let's look at some sample examples: 
\begin{enumerate}[noitemsep]
	\item ``I'm glad you came'' $\rightarrow$ \ita{Me alegro de que vinieras.} or \ita{Me alegro de que hayas venido.} \footnote{Another way of saying this could be with \ita{alegrar} instead of \ita{alegrarse de}.  Hence, additional translations would include \ita{Me alegra que hayas venido} and \ita{Me alegra que vinieras}.}
	\item ``They hate it when the park is crowded. '' or ``They hate it that the park is crowded.'' $\rightarrow$ \ita{Odian que el parque est\'e lleno} or \ita{Odian que el parque est\'a lleno}
	\item ``She's scared you'll fail the exam.'' $\rightarrow$ \ita{Teme que repruebes el examen.}
	\item ``I was so angry that he was ignoring us.'' $\ita{Estaba tan enfadada que nos ignorara.}$
	\item ``I'm so sorry that you were sick.'' $\rightarrow$ \ita{Lamento que estuvieras enfermo.}
\end{enumerate}

\begin{conf}{Rules? I don't know her.}
	This is the first (of five billion) instances where there are multiple right answers to a translation, where one translation uses the subjunctive, and the other uses the indicative. \underline{These statements are not synonymous.} While both are correct, they have different meanings. \\
	
\\	In some of these cases, we're required to use the subjunctive, and in others, we're required to use the indicative. In other cases, there aren't really any rules to guide us; the mood used depends on how factual/uncertain the speaker is about a statement. It's pretty important to realize which of the three scenarios apply to any given sentence, and the best way to get used to these kinds of things is to increase your exposure to the language. \\

\\ For this particular section, you'll need to use the subjunctive more often than not. For emotional expressions, the default usage is with the subjunctive, and if the speaker feels more certain, the mood can be changed from its default to the indicative. 
\end{conf}

Let's look at the first sentence, ``I'm glad you came.'' This one is pretty self-explanatory. 
\begin{enumerate}
	\item The first bit of the sentence becomes \ita{Me alegro de que}. 
	\item Since it's an emotional expression, we'll need to use the subjunctive in the subordinate clause, so the latter half of the sentence becomes \ita{vinieras} or \ita{hayas venido}. Here, we have a choice between the subjunctive imperfect and the present perfect, and you can pick whichever one you want. 
	\item Thus, the complete translation is \ita{Me alegro de que hayas venido} or \ita{Me alegro de que vinieras}.
\end{enumerate}

The second sentence gets a bit more fun, ``They hate it when the park is crowded.''
\begin{enumerate}
	\item First off, sentences like ``I like it when,'' ``We hate it when,'' ``It frustrates us when,'' all fall under emotional expressions in Spanish. Those three phrases can be translated as \ita{Me gusta que}, \ita{Odiamos que}, \ita{Nos frustra que}. Hence, the first bit of the translation becomes \ita{Odian que}.
	\item The second bit of the translation is also interesting. As we talked about earlier, this is one of those scenarios where you could technically use either mood. 
		\begin{itemize}
			\item If we use the subjunctive, it means that they just hate it when the park is crowded. It's a general statement, applying to whenever the park happens to be crowded. 
			\item If we use the indicative, it means that they hate it that the park is crowded, implying that the park is definitely crowded. This statement conveys more certainty, since it's said in reference to a definite event. 
		\end{itemize}
	\item Hence, we get two different translations, \ita{Odian que el parque est\'e lleno} or \ita{Odian que el parque est\'a lleno}. 
\end{enumerate}

The remaining example sentences follow the first example. Working them out is left as an exercise for the reader. 
\section{I: Impersonal Expressions}
%%%%%%%%%%%%%%%%%%%%%%%%%%%%%%%%%%%%%%%%%%%%%%%%%%%%%%%%%%%%%%%%%%%%%%%%%%%%%%%%%%%%%%%%%%%%%%%%%%%%%%%%%
This is one of the easiest uses of the subjunctive. \\

Impersonal expressions are statements that lack subjects. Here are a list of sample impersonal expressions along with their corresponding translations:
\begin{enumerate}[noitemsep]
	\item ``It's too hot today.'' $\rightarrow$ \ita{Hoy hace demasiado calor.}
	\item ``It's important to eat healthy and work out.'' $\rightarrow$ \ita{Es importante comer sano y hacer ejercicio}.
	\item ``It's important for you to eat healthy and work out.'' $\rightarrow$ \ita{Es importante que comas sano y que hagas ejercicio.}
	\item ``It's a pity that you were sick.'' $\rightarrow$ \ita{Es una l{\'a}stima que estuvieras enfermo.}
	\item ``It was the best of times, it was the worst of times ...'' $\rightarrow$ \ita{Era el mejor de los tiempos, era el peor de los tiempos ... } (Charles Dickens, \underline{A Tale of Two Cities})
\end{enumerate}

As you can see, some of those sentences used the subjunctive (the third and fourth statements), and the rest didn't. \\

How do we know if an impersonal expression will take the subjunctive or not? This bit is pretty simple. Are there two different subjects (one is null, the other is present)? If so, use the subjunctive. If not, just go with the infinitive. \\

One thing to keep in mind is that in English, we tend to use the preposition ``for,'' while in Spanish, in these sorts of situations, we use the relative pronoun \ita{que}. \\

With that in mind, let's take the third example in this section, the sentence ``It's important for you to eat healthy and work out.''
\begin{enumerate}
	\item In Spanish, the first bit of the translation is self-explanatory, we get \ita{Es importante}.
	\item Instead of the word ``for,'' we use \ita{que}. Since there's a definite subject associated with the second half of the sentence (you, or \ita{t\'u}), we conjugate the verbs in the present subjunctive.
	\item Hence, our final translation becomes \ita{Es importante que comas sano y que hagas ejercicio}. 
\end{enumerate}

Let's work through the fourth statement, ``It's a pity that you were sick.'' This follows similarly. 
\begin{enumerate}
	\item The first bit is translated as normal; ``It's a pity'' becomes \ita{Es una l\'astima}. Instead of the word ``for,'' we use \ita{que}. 
	\item Since there's a definite subject associated with the second half of the sentence (you, or \ita{t\'u}), we conjugate the verb. However, unlike the previous example, we need to conjugate our verb in the subjunctive imperfect. This is because the event of being sick occurred in the past (given the context of our sentence, we can assume that the person being addressed was sick in the past and is now okay).
	\item Hence, our final sentence becomes \ita{Es una l\'astima que estuvieras enfermo.}
\end{enumerate}

%%%%%%%%%%%%%%%%%%%%%%%%%%%%%%%%%%%%%%%%%%%%%%%%%%%%%%%%%%%%%%%%%%%%%%%%%%%%%%%%%%%%%%%%%%%5%%%%%%%%%%
\section{R: Recommendations}

The most common use of the subjunctive is in recommendations or suggestions. \\

If we want to tell someone to do something, we don't really have a guarantee that they'll actually do it. That's why the subjunctive is \underline{always} used in sentences with recommendations or suggestions, since we're not that sure about the outcome.\\

Let's start out by just listing some sample sentences with recommendations along with their corresponding translations:
\begin{enumerate}[noitemsep]
	\item ``My parents want me to plan more.'' $\rightarrow$ \ita{Mis padres quieren que planee m\'as.}
	\item ``I'd like for you to clean your room.'' $\rightarrow$ \ita{Me gustar\'ia que limpiaras tu cuarto.}
	\item ``The doctor advises you to avoid caffeine.'' $\rightarrow$ \ita{El doctor te aconseja que evites la cafe\'ina.}
	\item ``I already told you to start your homework!'' $\rightarrow$ \ita{\textexclamdown Ya te dije que empezaras tu tarea!}
	\item Like 50\% of the lyrics in Shakira's (feat. Alejandro Sanz) \ita{La Tortura}. Requests (\ita{pedir que + verbo}) of another person take the subjunctive.  \footnote{I highly recommend you to listen to this song to drill in the uses of the subjunctive with recommendations. Music in general is pretty good for drilling in concepts. }
\end{enumerate}

One thing to take note of is that in English, we tend to use the preposition ``to'' in these sorts of constructions. For example, we can say ``The doctor recommended him to avoid caffeine,'' as well as ``The doctor recommended that he avoid caffeine''; both are correct.\\ 

Additionally, in these constructions, it may be possible to use the infinitive and completely avoid having to conjugation in special cases, such as with the verbs \ita{hacer}, \ita{dejar} \ita{forzar}, \ita{mandar}, or \ita{permitir}, where it's actually more common to use the infinitive. For example, if we wanted to translate ``She let me borrow her pencil'' \footnote{Sentence said by a classmate in 9\textsuperscript{th} grade.}, we can just say \ita{Ella me dej\'o utilizar su l\'apiz}, instead of \ita{Ella dej\'o que utilizara su l\'apiz.}
%%%%%%%%%%%%%%%%%%%%%%%%%%%%%%%%%%%%%%%%%%%%%%%%%%%%%%%%%%%%%%%%%%%%%%%%%%%%%%%%%%%%%%%%%%%%%%%%%%%%%%
\section{D: Doubt \& Denial}

This is one of the more challenging uses of the subjunctive, right along with emotional expressions. We're going to break this up into two subsections, and go through each one. 

\subsection{Doubt}

The most self-explanatory use of the subjunctive is to express doubt, by definition, expresses uncertainty. However, it might be a bit more complicated than that, as we'll see later in the section. \\

The easiest way to talk about doubt with the subjunctive, is to go to verbs commonly used in this context: \ita{dudar} (to doubt) and \ita{negar} (to deny) are the two verbs that solely express doubt. For example, to translate the sentence ``I doubt she'll understand,'' we can say \ita{Dudo que entienda}. \\

An important thing to keep in mind is that when we add the word \ita{no} in front of these two verbs, we need to use the indicative, since saying that you're not doubting something is really an expression of certainty, thus warranting the use of the indicative.
\subsection{Denial (subset of doubt)}

Another way of expressing uncertainty is by negating a statement that would expressing, such as saying that you're not sure. Common expressions with this include \ita{no pensar/creer que}, \ita{no entender que} or \ita{no parecer que}. These statements often take the subjunctive. 

\subsection{Confusion :/}

So what's the confusion? It's just that some of the expressions I just mentioned \underline{can} be used with the indicative, or can be used with the subjunctive. Typically, what happens is that a certain expression \ita{tends} to use one mood and not the other, and if you use the other mood, it means that you really want to emphasize the certainty/uncertainty of an event happening. \\

For the beginner student, if you can read the rest of this paragraph, that'll suffice. However, if you're a more advanced student, you may want to read through the rest of this section. The rules we've laid out thus far are mostly all you'll need, at least in the short term. Basically, if it expresses certainty and/or is an affirmative sentence, use the indicative. If not, consider using the subjunctive. \\

For the advanced student, this bit exemplifies the beauty and nightmare in the Spanish subjunctive. In these cases, the subjunctive truly gives you a lot of liberty. Let's look at a couple of different translations of the sentence ``I admit that I (may) have hurt him,'' \ita{Admito que lo he herido} or \ita{Admito que lo haya herido.} The indicative is used if and only if the speaker is certain that the chap in question is hurt. The subjunctive is used if and only if the speaker is not sure if the chap is hurt. In English, we sometimes use the word ``may'' to express this kind of uncertainty, and can sometimes imply that we'll need to use the subjunctive in the translation.

\begin{conf}{Freedom!}
	While some of the other sections that we've looked at either clearly stated whether or not we should use the subjunctive, this bit is quite different. In this section, there aren't really any grammatical rules to guide us. The speaker can pretty much use whichever mood they want to use, depending on how certain they \underline{feel} about the event. We have a ton of freedom here, since in the vast majority of doubt and denial cases, using either the subjunctive or the indicative are both grammatically correct. Using the indicative or the subjunctive only modifies the meaning of the sentence, to convey how certain the speaker feels. 
\end{conf}

Let's go over another example for funsies. Let's translate the sentence ``I suspected he was the murderer.'' Again, there are a couple different translations we could use, depending on what we wanted to express: \ita{Sospechaba que era el asesino} or \ita{Sospechaba que fuera el asesino}. Here, if we use the form \ita{era}, \ita{sospechar} takes on a meaning closer to \ita{creer/pensar}, meaning that the speaker felt more certain that the individual in question was the murderer at that point in time. However, if we use the form \ita{fuera}, it means that the speaker felt more unsure than sure that the person was the murderer. 
%%%%%%%%%%%%%%%%%%%%%%%%%%%%%%%%%%%%%%%%%%%%%%%%%%%%%%%%%%%%%%%%%%%%%%%%%%%%%%%%%%%%%%%%%%%%%%%%%%%%%%
\section{O: \ita{Ojal{\'a}}}

This might be the coolest use of the subjunctive, and it's also probably the easiest. There's not much thinking involved, since the word \ita{ojal\'a} always triggers the subjunctive. \\

The word \ita{ojal\'a} comes from the Arabic expression \ita{'In sh{\=a}' All{\=a}h}, which can be translated as ``God willing.'' \ita{Ojal\'a} can mean ``if only,'' ``I wish,'' or ``hopefully.'' As we can see, all of those expressions convey a great deal of uncertainty, since we only use them with hypothetical situations or with events we're not sure will happen.\\

When we use \ita{ojal\'a} to mean ``if only'' or ``I wish,'' we need to conjugate our verb in the subjunctive imperfect, since these are hypothetical situations. Let's look at a few examples. \\

Let's translate the sentence ``I wish you were here,'' which can be expressed as \ita{Ojal\'a que estuvieras aqu\'i} or \ita{Ojal\'a estuvieras aqu\'i}. With the word \ita{ojal\'a}, you can either just use \ita{ojal\'a} or you can say \ita{ojal\'a que}. The choice is left up to the speaker's discretion. Personally, I prefer saying \ita{ojal\'a que}, so that's what I'm going to use for the rest of the guide. Since we want to express the idea of ``I wish,'' we need to conjugate the verb \ita{estar} (to be) in the subjunctive imperfect, so we get \ita{estuvieras}. As our final translation, we get \ita{Ojal\'a que estuvieras aqu\'i}. Pretty easy, right? \\

Let's now do an example with an ``if only'' situation; let's translate ``If only they had listened.'' Again, since we want to convey the idea of ``if only'' using the word \ita{ojal\'a}, we'll need to conjugate the verb \ita{escuchar} (to listen) in the subjunctive imperfect, to get \ita{escucharan}. Hence, as our final translation, we get \ita{Ojal\'a que escucharan.} \\

Like we talked about earlier, \ita{ojal\'a} can also be used to mean ``hopefully.'' When we use it in that context, we conjugate our verb in the present subjunctive. Let's work through an example like that. \\

Let's translate the sentence ``Hopefully it'll be sunny tomorrow.'' Since we're using \ita{ojal\'a} to mean ``hopefully,'' we need to conjugate the verb/expression \ita{hacer sol} in the present subjunctive, to get \ita{haga sol}. Hence, our final translation is \ita{Ojal\'a que haga sol ma\nye ana}.\\

That's all for now! In the next chapter, there are exercises for you to practice using the subjunctive with WEIRDO clauses and to build up some confidence. After the practice exercises, we'll continue onto cases where we use the subjunctive, but are \underline{not} linked to WEIRDO clauses. 
