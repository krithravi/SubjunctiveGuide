What is the subjunctive? \\

It's a mood, so it reflects the speaker's ... mood. In Spanish, there are three different moods: the indicative, the subjunctive, and the imperative.
\begin{conf}{Mood versus Tense}
 A super common mistake is to call the subjunctive a tense, but a tense is only used to refer to time. Yes, you'll hear instructors sometimes refer to a ``subjunctive tense,'' in which case they're probably referring to something called the present subjunctive. Yes, they're wrong, but it's just a technicality. 
\end{conf}
The indicative mood probably comprises most of what you've seen so far. It's used for when the speaker is pretty certain about what they're talking about. For instance, if I were to say ``It's going to rain tomorrow,'' that would imply that I'm pretty sure it's going to rain. Hence, we'd translate that as \textit{Va a llover ma{\nye}ana} or \textit{Llover\'{a} ma{\nye}ana}. \\

The imperative mood is used to tell someone to do something, like a command or an order. For example, to tell your friend to ``hurry up,'' we would use the imperative mood. That sentence would translate to \textit{{\exc}Ap\'{u}rate ya!}, \textit{{\exc}Ten prisa!}, or \textit{{\exc}Date prisa!}. \\

Now, let's talk about the subjunctive mood. The subjunctive mood is the \underline{exact} opposite of the indicative mood. While the indicative mood was used to express certainty, the subjunctive mood is used to express uncertainty. For instance, to say that you wished you had studied harder, you would need to use the subjunctive. The sentence \textit{I wish I had studied harder} can be translated as \textit{Ojal\'{a} que hubiera estudiado m\'{a}s}. Notice that we didn't use the indicative and say \textit{hab\'{i}a estudiado}. \\

We'll talk more about when to use the subjunctive in the next chapters, but mostly in the \hyperref[sec:weirdo]{WEIRDO Clauses} section and in the \hyperref[sec:other]{Other Phrases} sections.\\

There's a list of tenses and moods in the \hyperref[subsec:tense]{List of Tenses and Moods} section in the Appendix. If the stuff that we've gone over so far is 100\% new to you, I recommend taking a look at that list before we move on. \\

Let's talk about how to conjugate in the subjunctive in the next chapter. 





